% $Id: update.tex,v 2.0 2001/09/14 19:05:06 garyb Exp $
\documentstyle[12pt]{article}
\textwidth=7.0in
\oddsidemargin=-0.25in
\begin{document}
\begin{center}

\bf Update to Bernstein-Khushalani Orbit-Fitting Software

\em G. Bernstein, $Date: 2001/09/14 19:05:06 $

\end{center}


The orbit-fitting software uses the methods described by Bernstein \&
Khushalani 2000, ``Orbit Fitting and Uncertainties for
Kuiper Belt Objects,'' {\em AJ} {\bf 120} 3323.  I have since refined
the algorithms used by the software to better recognize and fit
degenerate orbits, and to incorporate features such as orbiting
observatories. This document describes the updated algorithms, and
describes the other conveniences and features added to the software
since its original release to the public.  I assume that you've read
the original paper before reading this document.  The {\tt README}
file with the software is a more complete guide to installing and
running the software.

{\em Please remember that the software calculates {\bf barycentric}
osculating elements, not the more commonly used heliocentric
elements.}  The barycentric elements are more stable for KBOs, but
disagree with the MPC's conventions.

\section{Algorithms}
\subsection{Recognizing and Fitting Degenerate Orbits}
In the original code, the steps for recognizing and fitting degenerate
orbits were as follows:
\begin{enumerate}
\item Do a preliminary linearized 5-parameter fit to the observations,
with $\dot\gamma$ set to zero.  This step was skipped if the number of
observations is only 2, since the fit would be degenerate.
\item Use the linear fit as a starting point for a full non-linear
6-parameter fit.
\item If the 6-parameter fit has a large ${\rm Var}(\dot\gamma),$ then
set $\dot\gamma=0$ and execute a non-linear fit to the other 5
parameters, placing a nominal value for ${\rm Var}(\dot\gamma)$ into
the output covariance matrix $\Sigma$.  
\item If the 5-parameter fit still has unbound orbits within the
uncertainty range, fit again but add an energy constraint to the
$\chi^2$ being minimized, reducing the degrees of freedom by one.
\end{enumerate}

I have realized that Step (3) of this process can lead in some cases
to orbit fits that are misleadingly precise.  This occurs when the
observations are very accurate in constraining 5 degrees of freedom,
but still have a strong degeneracy.  Examples are:  having only two
observation nights but with several months between them; HST
observations, with highly precise astrometry but over a short arc (few
days); or two or more observations, but all at nearly the same time of
year, so that line-of-sight motion remains indeterminate.

In these cases the degeneracy is primarily, {\em but not exactly,} along
$\dot\gamma$.  Then when we set $\dot\gamma=0$, we force the solution
to a very precise location in the six-dimensional space and appear to
get a very precise orbit, when in fact it is strongly degenerate.

To remedy this situation, and to more clearly recognize when the orbit
is doubly degenerate, the procedure is now as follows:
\begin{enumerate}
\item  Do a preliminary linearized 5-parameter fit to the observations,
with $\dot\gamma$ set to zero.  The uncertainty on $\gamma,$ {\it
i.e.} the KBO distance, is examined, and if
\begin{equation}
{\sigma_\gamma \over \gamma} > 0.2
\end{equation}
then we immediately recognize that the arc is too short to constrain
even 5 parameters, and the orbit is doubly degenerate.
\item If we pass the above test, then we execute the full
6-dimensional fit, as before.
\item If the 6-parameter fit has ${\rm Var}(\dot\gamma)$ comparable to
the velocity needed to unbind the orbit, then we recognize the orbit
as singly degenerate.  In this case, the 6-parameter fit is repeated,
but with a constraint on energy so that we are minimizing a
pseudo-$\chi^2$ 
\begin{equation}
\tilde\chi^2 \equiv \sum_i \left[ { (\hat\theta_{x,i} - \theta_{x,i})^2 \over
\sigma_i^2 } + { (\hat\theta_{y,i} - \theta_{y,i})^2 \over
\sigma_i^2 } \right] + { f_b^2  \over 3},
\end{equation}
where the binding energy fraction $f_b$ is defined by
\begin{equation}
\dot\alpha^2 + \dot\beta^2 + \dot\gamma^2 =
(1+f_b) GM_\odot \gamma^3.
\end{equation}
Again, $f_b=0$ corresponds to a circular orbit, and $-1<f_b<1$ for
bound orbits.  Hence the addition of the energy constraint pushes the
orbit to be bound, and close to circular.
\item If there were only 2 observations, or if Step (1) indicated a
doubly degenerate orbit, then the $\tilde\chi^2$ minimization is done
with $\dot\gamma$ fixed at zero, with a nominal variance placed on
$\dot\gamma$ post-facto.
\end{enumerate}
Hence the primary change is that we use the energy constraint, rather
than setting $\dot\gamma=0$, as a means of handling singly-degenerate
orbits.  This avoids the (sometimes erroneous) assumption that {\em
all} of the degeneracy is in $\dot\gamma$.  

Another bug which was found to arise is that the traditional means of
minimizing $\chi^2$ and calculating $\Sigma$ is to ignore the second
derivatives of the fitted quantities, as is done in the standard {\it
Numerical Recipes} routines that I used.  This approximation turns out
to often be a poor one for the $f_b$ term in our $\tilde\chi^2$.  The
code now explicitly inserts the necessary second-derivative terms.
This bug had caused unrealistically small error bars to be placed on
some very degenerate orbits.

The default error placed on $\dot\gamma$ for doubly-degenerate orbits,
and the strength of the constraint on $f_b$, are weaker than in the
last release of the software, meaning the degenerate orbits will show
somewhat larger errors in position and elements than before.

\subsection{Orbiting Observatories}
In preparing a proposal for KBO observations with HST, I produced code
to compute positions from observatories in orbit around the Earth.  If
an observation uses an observatory code which is $\ge2000$, it is
assumed to be an orbiting observatory.  The {\tt observatories.dat}
file must containing a corresponding entry with the following data on
a line:

\noindent
{\it obscode   inclination  period  precession\_period  jd0  RA0  name}

\noindent
with the inclination (degrees) and period (days) of the spacecraft
orbit, the precession period (in days) of the orbit pole
about the Earth pole, the JD at which the spacecraft is at the
ascending node of the orbit with pole directed toward a particular
RA.  The supplied {\tt observatories.dat} file contains an entry which
allows simulation of the HST orbit.  This is a schematic
representation of the orbit, of course, just simulating the motions
and seeing the effect on orbital fits, parallax, etc.
Precise actual data must know the phase of the HST orbit more
carefully than this simple parameterization can provide.  Indeed due
to variable drag it is impossible to predict the HST position to
better than a few thousand km even a month in advance.

The spacecraft orbit is taken to be circular.

\section{New Conveniences}
The basic programs now accept command-line arguments through which one
can specify the location of the binary JPL ephemeris and/or the
observatories database file.  Also the uncertainty assigned to
observations in MPC format can be altered on the command line.

Alternatively the locations of the ephemeris and observatory files can
be specified by setting the environment variables {\tt ORBIT\_EPHEMERIS}
and {\tt ORBIT\_OBSERVATORIES}.  For example your {\tt .cshrc} file
could contain the line

\noindent
{\tt setenv ORBIT\_OBSERVATORIES /usr/local/orbits/observatories.dat}

\noindent
If neither of these is done, the programs look in the current
directory as before.

The Release code for the software can be found by putting the {\tt -v}
option on the command line of most of the basic programs as well.

Sample input and output files for Pluto and 1992~QB$_1$ are now
included in the distribution, to help check for proper compilation.

\section{New Subroutines}
The {\tt ephem\_earth.c} code has been reorganized to handle the
orbiting observatories described above.  Some additional subroutines
have been created, however, which are useful in creating your own
programs: 
\begin{itemize}
\item {\tt elements\_to\_xv()} will create a Cartesian-basis orbit
representation from an orbital-element representation; previously,
only the reverse transformation was available.
\item {\tt zenith\_angle()} will calculate the angle between an
observation direction and the anti-geocenter vector.
\item {\tt zenith\_horizon()} will calculate the angle from zenith to
horizon for a chosen observatory.  A spherical Earth is assumed for
orbiting observatories, while ground-based observatories just return
90 degrees.
\item {\tt is\_visible()} tells whether a given observation is above
the horizon.
\end{itemize}


Other changes to the code since the first release include the addition
of more observatories to the included {\tt observatories.dat} file.  I
believe the file currently contains all sites that have contributed
KBO observations at $R>21$ to the MPC.

The package is now under CVS control, so each file has a header that
gives the version number.  If you find bugs, etc., please let me know
the version information of the files containing the bugs.
\end{document}
